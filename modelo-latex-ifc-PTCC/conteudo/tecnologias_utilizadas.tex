\section{Tecnologias Utilizadas}

Para o desenvolvimento da interface do sistema de gestão de retiros espirituais, está sendo proposto um conjunto de tecnologias modernas, escolhidas com o objetivo de garantir desempenho, escalabilidade, facilidade de manutenção e uma experiência intuitiva para os usuários. A seguir, apresentam-se as tecnologias selecionadas e a justificativa para a adoção de cada uma, considerando como elas contribuirão para o projeto.

\subsection{Desenvolvimento Front-end}

O front-end será desenvolvido utilizando a linguagem TypeScript, uma extensão do JavaScript que adiciona tipagem estática ao código. Essa abordagem proporcionará maior segurança durante o desenvolvimento, auxiliando na detecção precoce de erros relacionados a tipos de dados, além de tornar o código mais legível, estruturado e fácil de manter em médio e longo prazo.

Além disso, será utilizado o framework Next.js, baseado em React. O Next.js oferece recursos de renderização híbrida (tanto server-side quanto client-side), permitindo a construção de páginas mais rápidas e com melhor desempenho, o que contribuirá significativamente para a experiência do usuário. A estrutura do framework também facilitará o roteamento automático entre páginas, promovendo um desenvolvimento ágil e uma navegação fluida.

Para a construção da interface visual, será adotada a biblioteca de componentes Material UI (MUI), fundamentada nas diretrizes do Material Design desenvolvido pelo Google. Essa biblioteca agiliza a criação de interfaces responsivas e acessíveis, assegurando consistência visual e uma boa usabilidade em diferentes dispositivos — aspecto essencial, dado o público diversificado do sistema de retiros espirituais.

\subsection{Controle de Versão}

O gerenciamento do código-fonte será realizado por meio do sistema de controle de versão Git, tecnologia que facilitará a colaboração entre os membros da equipe. O uso do Git permitirá o controle detalhado das alterações no código, a criação de branches para novas funcionalidades e a resolução eficiente de conflitos durante o desenvolvimento.

Complementarmente, será utilizada a plataforma GitHub para hospedagem do repositório remoto do projeto. A escolha da plataforma justifica-se pela integração facilitada com práticas de desenvolvimento ágil, como o acompanhamento de tarefas por meio de issues, a revisão colaborativa por pull requests e a automação de processos através de pipelines de integração contínua (CI/CD).

\subsection{Testes}

Com o intuito de assegurar a qualidade do sistema e o funcionamento correto de suas funcionalidades, será planejada a realização de testes unitários e de integração. As ferramentas Jest e Testing Library serão empregadas nesse processo. O Jest possibilitará a execução de testes automatizados de forma rápida e confiável, enquanto a Testing Library permitirá a simulação de interações reais dos usuários com os componentes da interface, garantindo que o sistema reaja adequadamente às ações esperadas.

\subsection{Implantação (Deploy)}

A implantação da aplicação será realizada por meio da plataforma Vercel, especializada em hospedagem de aplicações front-end. A escolha dessa plataforma se deve à sua integração direta com o GitHub, o que permitirá um fluxo contínuo e automatizado de deploy: cada alteração no repositório será capaz de disparar automaticamente a publicação da versão mais recente da aplicação.

O sistema será acessado via web, e o design será desenvolvido com foco em responsividade, visando proporcionar uma experiência satisfatória de uso tanto em computadores quanto em dispositivos móveis, como tablets e smartphones. Dessa forma, espera-se atender às diversas necessidades e contextos de uso dos participantes e organizadores dos retiros.