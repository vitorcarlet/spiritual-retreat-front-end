\chapter{Metodologia}

A metodologia que será adotada neste trabalho tem como objetivo estruturar de forma clara e sistemática o processo de desenvolvimento do front-end do sistema SAMGestor, voltado à gestão de retiros espirituais. A abordagem metodológica orientará cada etapa da construção da interface da aplicação, garantindo coerência com os objetivos da pesquisa, bem como a reprodutibilidade e a confiabilidade dos resultados que se pretende alcançar.

A seguir, serão descritos os principais componentes metodológicos propostos:

\section{Abordagem da Pesquisa}

Esta pesquisa caracterizar-se-á como aplicada e de natureza qualitativa, com enfoque exploratório-descritivo. A abordagem aplicada justificar-se-á pela finalidade prática do projeto, que buscará oferecer uma solução concreta para um problema real enfrentado por organizadores de retiros. O caráter qualitativo manifestar-se-á na análise de requisitos, na construção iterativa das interfaces e na avaliação da experiência do usuário, considerando aspectos subjetivos e contextuais do uso do sistema.

\section{Instrumentos de Coleta de Dados}

Serão utilizados como instrumentos principais: levantamento de requisitos junto à equipe organizadora do retiro, análise documental de processos existentes (como planilhas, listas de inscrições e comunicados internos), bem como validações contínuas por meio de feedbacks durante o desenvolvimento. A coleta de dados também considerará diretrizes de usabilidade e acessibilidade baseadas em boas práticas de design de interface.

\section{Procedimentos de Coleta de Dados}

A coleta de dados será realizada de forma colaborativa e contínua ao longo do projeto, por meio de reuniões periódicas com os \textit{stakeholders}, com o intuito de levantar necessidades, validar funcionalidades e aprimorar fluxos da interface. Essas interações permitirão compreender os papéis dos usuários no sistema (administradores, coordenadores, participantes) e mapear os principais módulos do sistema, como controle de inscrições, barracas, famílias, equipes, dashboards e notificações.

Como parte do processo de concepção visual e funcional, serão elaborados protótipos navegáveis utilizando a ferramenta Figma, os quais servirão de base para a validação antecipada da experiência do usuário e para orientar o desenvolvimento do front-end.

\section{Procedimentos de Análise de Dados}

Os dados coletados serão analisados com base na técnica de análise de conteúdo, a fim de identificar padrões, necessidades recorrentes e pontos críticos nos processos atuais de gestão do retiro. A partir dessa análise, serão definidos os requisitos funcionais e não funcionais do sistema, os quais posteriormente serão transformados em protótipos e componentes funcionais utilizando tecnologias como React, Next.js, Material UI e TypeScript.

\section{Considerações Éticas}

Todas as informações obtidas junto aos organizadores serão utilizadas exclusivamente para fins acadêmicos, com o devido consentimento e garantia de confidencialidade. Nenhum dado sensível será exposto, armazenado ou compartilhado sem autorização expressa dos envolvidos. O projeto respeitará os princípios éticos da pesquisa científica, conforme as diretrizes da instituição de ensino.