\section{Requisitos Funcionais}

A tabela a seguir apresenta os requisitos funcionais identificados para o desenvolvimento do front-end do sistema SAMGestor. Eles descrevem o comportamento esperado da interface em diferentes situações e perfis de uso.


\begin{longtable}{|p{4cm}|>{\raggedright\arraybackslash}p{\dimexpr\linewidth-4.2cm}|}
\caption{Requisitos Funcionais} \label{tab:req-funcionais} \\
\hline
\textbf{Código} & \textbf{Requisito Funcional} \\
\hline
\endfirsthead
\caption{Requisitos Funcionais} \label{tab:req-funcionais} \\
\hline
\textbf{Código} & \textbf{Requisito Funcional} \\
\hline
\endfirsthead

\multicolumn{2}{c}%
{{\bfseries \tablename\ \thetable{} -- Continuação da página anterior}} \\
\hline
\textbf{Código} & \textbf{Requisito Funcional} \\
\hline
\endhead

\hline \multicolumn{2}{r}{{Continua na próxima página}} \\
\endfoot

RF01 & A interface deve exibir menus, botões e seções com base no perfil do usuário logado (Administrador, Gestor ou Consultor). Elementos não permitidos devem ser ocultados ou desativados. \\
\hline
RF02 & A interface deve detectar inatividade e redirecionar para a tela de login após X minutos, exibindo mensagem de inatividade. \\
\hline
RF03 & Administradores devem visualizar todos os módulos, incluindo gestão de usuários e atribuição de papéis, com botões de criar, editar e excluir sempre disponíveis. \\
\hline
RF04 & Gestores devem visualizar apenas funcionalidades relacionadas à gestão de retiros, inscrições, famílias, equipes e barracas. Módulos de usuários devem ser ocultos. \\
\hline
RF05 & Consultores devem ter acesso somente de leitura, com todos os botões de ação desativados ou ocultos. \\
\hline
RF06 & Exibir formulário para criação/edição de retiros com campos obrigatórios como nome, tema, período, vagas e taxas. \\
\hline
RF07 & Impedir envio do formulário se campos obrigatórios não forem preenchidos, com mensagens de erro específicas. \\
\hline
RF08 & Exibir feedback visual (mensagem ou destaque) parNãostro no sistema. \\
\hline
RF11 & Atualizar automaticamente os dados na interface após alterações. \\
\hline
RF12 & Disponibilizar formulário público de inscrição com campos obrigatórios como Nome, CPF, Nascimento e Cidade. \\
\hline
RF13 & Validar CPF e Data de Nascimento com mensagens específicas em caso de erro. \\
\hline
RF14 & Caso inscrições estejam encerradas, bloquear o formulário e exibir mensagem ou opção de acesso por senha. \\
\hline
RF15 & Gestores devem ter formulário semelhante ao público com opção de importação via .csv e feedback da operação. \\
\hline
RF16 & Exibir status da inscrição com rótulos visuais: Não Contemplada, Contemplada, Confirmada, Pendente, Cancelada. \\
\hline
RF17 & Permitir que CPF seja marcado como “Bloqueado” e impedir nova inscrição com aviso. \\
\hline
RF18 & Após contemplação, permitir envio de mensagens automáticas aos não contemplados. \\
\hline
RF19 & Exibir tabela com filtros para grande volume de dados e destacar colunas. \\
\hline
RF20 & Exibir percentual de ocupação de vagas por gênero em tempo real. \\
\hline
RF21 & Permitir seleção múltipla de inscritos para contemplação (inclusive por filtros como cidade/região). \\
\hline
RF22 & Após contemplação, permitir envio de mensagens individuais ou em lote. \\
\hline
RF23 & Alertar e impedir contemplação de CPF já marcado como “FEITO” ou “Bloqueado”. \\
\hline
RF24 & Permitir ao gestor atualizar manualmente o status de confirmação. \\
\hline
RF25 & Após confirmação, alterar status para “Pagamento Pendente” automaticamente. \\
\hline
RF26 & Apresentar opções de pagamento (Pix, Boleto, Cartão) após confirmação de participação. \\
\hline
RF27 & Exibir status de pagamento com rótulos distintos: Pendente, Pago, Cancelado. \\
\hline
RF28 & Exibir formulário de cadastro de famílias com validações. \\
\hline
RF29 & Permitir movimentação de membros entre famílias com drag-and-drop. \\
\hline
RF30 & Verificar em tempo real a capacidade máxima da família ao mover membros. \\
\hline
RF31 & Impedir alocação de parentes/cônjuges na mesma família com alerta informativo. \\
\hline
RF32 & Exibir porcentagem de homens e mulheres na família com destaque visual se estiver desequilibrado. \\
\hline
RF33 & Alertar se família não tiver ao menos dois padrinhos e duas madrinhas. \\
\hline
RF34 & Exibir botão “SORTEAR” para distribuir automaticamente os contemplados. \\
\hline
RF35 & Permitir ajustes manuais após sorteio, caso haja parentes/cônjuges juntos. \\
\hline
RF36 & Exibir lista de equipes padrão como "Casa da Mãe", "Casa do Pai", "Tapera". \\
\hline
RF37 & Permitir criação de equipes personalizadas pelo gestor. \\
\hline
RF38 & Permitir selecionar/editar a função do participante na equipe: Coordenador, Vice, Membro. \\
\hline
RF39 & Exibir contadores de membros e vagas disponíveis por equipe. \\
\hline
RF40 & Impedir adição de membros em equipe lotada com mensagem de aviso. \\
\hline
RF41 & Permitir realocação de participantes entre equipes via drag-and-drop ou seleção. \\
\hline
RF42 & Permitir cadastro em lote de barracas por intervalo numérico. \\
\hline
RF43 & Permitir criação de barraca individual com campos obrigatórios. \\
\hline
RF44 & Obrigatoriedade de definir categoria (Masculina ou Feminina) no cadastro da barraca. \\
\hline
RF45 & Bloquear edição de barraca lotada. \\
\hline
RF46 & Permitir definir hospedagem de inscritos por barraca na tela de contemplação/acomodação. \\
\hline
RF47 & Impedir alocação de inscrito em barraca lotada com mensagem informativa. \\
\hline
RF48 & Permitir geração de relatórios diversos: Contemplados, Famílias, Barracas, Camisetas, Equipes. \\
\hline
RF49 & Relatórios devem permitir filtros por data, status, família, etc., com atualização dinâmica. \\
\hline
RF50 & Apresentar relatórios em tabelas responsivas, com exportação para PDF ou planilhas. \\
\hline
RF51 & Exibir no dashboard total de participantes confirmados com pagamento. \\
\hline
RF52 & Exibir gráfico de participantes com pagamento confirmado vs. pendente. \\
\hline
RF53 & Exibir gráfico ou contador de percentual de homens e mulheres inscritos. \\
\hline
RF54 & Mostrar quantidade de membros por família com gráfico de barras. \\
\hline
RF55 & Dashboard deve usar gráficos simples (pizza, barras) e contadores destacados. \\
\hline
\end{longtable}

