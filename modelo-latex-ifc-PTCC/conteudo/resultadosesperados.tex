\chapter{RESULTADOS ESPERADOS}

Com a implementação do front-end do sistema de gestão de retiros espirituais SAMGestor, espera-se criar uma experiência de interface de usuário altamente interativa que atenderá a todos os processos da organização e execução dessas atividades de forma fluida e performática. Os resultados esperados abrangem aspectos técnicos, operacionais e de experiência do usuário, conforme detalhado a seguir.

\section*{1. Melhoria na Eficiência Operacional}

Prevê-se uma redução significativa no tempo gasto com tarefas administrativas e repetitivas, como gerenciamento da equipe gestora, inscrição de participantes, controle de pagamentos e formação de famílias e equipes. O sistema proverá apenas as ações cabíveis a cada usuário, conforme seu nível de permissão, e automatizará grande parte desses processos, permitindo que os organizadores concentrem seus esforços em atividades estratégicas e de suporte direto aos participantes~\cite{sommerville2011}.

\section*{2. Redução de Erros e Retrabalhos}

Com a eliminação do uso intensivo de planilhas, processos manuais e maior controle de permissões por parte da gestão, espera-se reduzir drasticamente a incidência de erros humanos. Isso inclui problemas com informações duplicadas, perda de dados e falhas na comunicação interna, contribuindo para uma gestão mais segura e confiável dos dados~\cite{pressman2016}.

\section*{3. Aumento da Transparência e Comunicação}

A implementação de notificações em tempo real por meio de tecnologias como Server-Sent Events (SSE) garantirá que os gestores e organizadores estejam sempre atualizados sobre o andamento das atividades, como confirmação de pagamentos e status de equipes. Isso promoverá uma comunicação mais fluida e transparente entre todos os envolvidos.

\section*{4. Experiência do Usuário Aprimorada}

Utilizando tecnologias modernas como Next.js e Material UI, o sistema proporcionará uma interface intuitiva, responsiva e acessível. Espera-se que isso resulte em uma experiência positiva para usuários com diversos níveis de habilidade tecnológica, aumentando o engajamento e a satisfação dos participantes e administradores do retiro~\cite{preece2013}.

\section*{5. Acesso Facilitado e Gestão Remota}

O sistema permitirá acesso remoto às informações e funcionalidades essenciais por meio de qualquer dispositivo conectado à internet. Isso facilitará a gestão em tempo real, especialmente durante os retiros, quando os responsáveis precisam acessar rapidamente informações críticas e tomar decisões assertivas~\cite{laudon2020}.

\section*{6. Escalabilidade e Manutenibilidade}

Com o uso de uma arquitetura modular baseada em componentes, espera-se garantir que futuras expansões e atualizações possam ser realizadas com facilidade, minimizando o tempo e o esforço necessário para adaptações e melhorias contínuas no sistema~\cite{gamma1995}.

\vspace{0.5cm}

Dessa forma, espera-se que o desenvolvimento do front-end do sistema SAMGestor contribua significativamente para uma gestão mais organizada, eficaz e alinhada às necessidades dos retiros espirituais, proporcionando benefícios diretos aos organizadores e participantes desses eventos.
