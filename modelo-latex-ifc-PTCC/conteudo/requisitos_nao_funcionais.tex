\section{Requisitos Não Funcionais}

Nesta seção, são descritos os requisitos não funcionais que orientam aspectos técnicos e de usabilidade da aplicação. Tais requisitos não dizem respeito diretamente às funcionalidades do sistema, mas impactam significativamente na qualidade da experiência do usuário.

\begin{longtable}{|p{4cm}|>{\raggedright\arraybackslash}p{\dimexpr\linewidth-4.2cm}|}
    \caption{Requisitos Não Funcionais} \label{tab:req-nao-funcionais} \\
    \hline
    \textbf{Código} & \textbf{Requisitos Não Funcionail} \\
    \hline
    \endfirsthead
    \caption{Requisitos Não Funcionais} \label{tab:req-nao-funcionais} \\
    \hline
    \textbf{Código} & \textbf{Requisitos Não Funcionail} \\
    \hline
    \endfirsthead
    
    \multicolumn{2}{c}%
    {{\bfseries \tablename\ \thetable{} -- Continuação da página anterior}} \\
    \hline
    \textbf{Código} & \textbf{Requisitos Não Funcionail} \\
    \hline
    \endhead
    
    \endfoot
RNF01 & A interface deve ser responsiva, adaptando-se a diferentes tamanhos de tela (desktop, tablet, smartphone). \\
\hline
RNF02 & A aplicação deve garantir uma experiência de usuário intuitiva e com tempo de resposta inferior a 2 segundos em ações comuns. \\
\hline
RNF03 & A interface deve seguir boas práticas de acessibilidade, como contraste adequado e navegação por teclado. \\
\hline
RNF04 & O sistema deve apresentar mensagens de erro claras e objetivas em todas as ações críticas. \\
\hline
RNF05 & A interface deve suportar o uso de componentes reutilizáveis com estados visuais consistentes. \\
\hline
RNF06 & O sistema deve ser compatível com os principais navegadores modernos (Chrome, Firefox, Edge). \\
\hline
\end{longtable}

