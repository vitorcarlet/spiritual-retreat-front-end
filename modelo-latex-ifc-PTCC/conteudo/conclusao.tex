\chapter{CONSIDERAÇÕES FINAIS}

Este projeto propôs o desenvolvimento da interface front-end do sistema SAMGestor, voltado à gestão de retiros espirituais realizados pela Comunidade Servos Adoradores da Misericórdia. A partir da análise do problema, da justificativa e da fundamentação teórica, foi possível delinear uma proposta de solução alinhada às necessidades de organização, controle e usabilidade dos envolvidos na realização desses eventos.

Foram definidos os objetivos gerais e específicos, estruturada a metodologia de desenvolvimento e apresentados os requisitos funcionais e não funcionais da aplicação. Além disso, foram descritas as regras de negócio e os principais fluxos da interface, acompanhados por protótipos desenvolvidos no Figma e diagramas de sequência que evidenciam a dinâmica do sistema.

O desenvolvimento do front-end será fundamentado em tecnologias modernas, priorizando acessibilidade, responsividade e facilidade de uso. A modelagem apresentada servirá como base para a implementação futura, a ser realizada na segunda etapa deste trabalho, o TCC2, que contemplará a construção prática da aplicação, testes e validações com usuários.

Com isso, espera-se que o sistema contribua de maneira significativa para a automação e eficiência dos processos relacionados à organização dos retiros, reduzindo o esforço manual, melhorando a comunicação com os participantes e oferecendo uma experiência de uso intuitiva e confiável para os organizadores.

